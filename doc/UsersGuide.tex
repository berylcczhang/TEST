\documentclass[12pt]{article}

\usepackage{hyperref, multirow}

\setlength{\oddsidemargin}{0in}
\setlength{\evensidemargin}{0in}
\setlength{\textwidth}{6.5in}
\setlength{\textheight}{8.5in}
\setlength{\topmargin}{-0.5in}
\setlength{\parindent}{0.5in}

\newcommand{\slug}{\texttt{SLUG}}

\begin{document}

\title{User's Guide for \slug\ v.~2.0}
\author{Mark Krumholz}

\maketitle

\tableofcontents

\clearpage

\section{License and Citations}

This is a guide for users of the \slug\ software package. \slug\ is distributed under the terms of the \href{http://www.gnu.org/licenses/gpl.html}{GNU General Public License v.~3}. A copy of the license notification is included in the main \slug\ directory. If you use \slug\ in any published work, please cite the \slug\ method paper, da Silva, R.~L., Fumagalli, M., \& Krumholz, M.~R., 2012, \textit{The Astrophysical Journal}, 745, 145. A second method paper, describing the upgraded version 2 code and a set of ancillary tools, is in preparation at this time.

\section{What Does \slug\ Do?}

\slug\ is a stellar population synthesis (SPS) code, meaning that, for a specified stellar initial mass function (IMF), star formation history (SFH), cluster mass function (CMF), and cluster lifetime function (CLF), it predicts the spectra and photometry of both individual star clusters and the galaxies (or sub-regions of galaxies) that contain them. In this regard, \slug\ operates much like any other SPS code. The main difference is that \slug\ regards the IMF, SFH, CMF, and CLF as probability distributions, and the resulting stellar population as being the result of a draw from them. \slug\ performs a Monte Carlo simulation to determine the PDF of the light produced by the stellar populations that are drawn from these distributions. The remainder of this section briefly describes the major conceptual pieces of a \slug\ simulation. For a more detailed description, readers are referred to \href{http://adsabs.harvard.edu/abs/2012ApJ...745..145D}{da Silva, Fumagalli, \& Krumholz (2012)}.

\subsection{Cluster Simulations and Galaxy Simulations}

\slug\ can simulate either a simple stellar population (i.e., a group of stars all born at one time) or a composite stellar population, consisting of stars born at a distribution of times. We refer to the former case as a ``cluster" simulation, and the latter as a ``galaxy" simulation, since one can be thought of as approximating the behavior of a single star cluster, and the other as approximating a whole galaxy.

\subsection{Probability Distribution Functions: the IMF, SFH, CMF, and CLF}

As mentioned above, \slug\ regards the IMF, SFH, CMF, and CLF as probability distribution functions. These PDFs can be described by a very wide range of possible functional forms; see Section \ref{sec:pdfs} for details on the exact functional forms allowed, and on how they can be specified in the code. When \slug\ runs a cluster simulation, it draws stars from the specified IMF in an attempt to produce a cluster of a user-specified total mass. There are a number of possible methods for performing such mass-limited sampling, and \slug\ gives the user a wide menu of options; see Section \ref{sec:pdfs}. 

For a galaxy simulation, the procedure involves one extra step. In this case, \slug\ assumes that some fraction $f_c$ of the stars in the galaxy are born in star clusters, which, for the purposes of \slug\, means that they all share the same birth time. The remaining fraction $1-f_c$ of stars are field stars. When a galaxy simulation is run, \slug\ determines the total mass of stars $M_*$ that should have formed since the start of the simulation (or since the last output, if more than one output is requested) from the star formation history, and then draws field stars and star clusters in an attempt to produce masses $(1-f_c)M_*$ and $f_c M_*$. For the field stars, the stellar masses are drawn from the IMF, in a process completely analogous to the cluster case. For star clusters, the masses of the clusters are drawn from the CMF, and each cluster is then populated from the IMF as in the cluster case. For both the field stars and the star clusters, the time of their birth is drawn from the PDF describing the SFH.

Finally, star clusters can be disrupted independent of the fate of their parent stars. When each cluster is formed, it is assigned a lifetime drawn from the CLF. Once that time has passed, the cluster ceases to be entered in the lists of individual cluster spectra and photometry (see next section), although the individual stars continue to contribute to the integrated light of the galaxy.

\subsection{From Stars to Light}

Once \slug\ has drawn a population of stars, its final step is to compute the light they produce. \slug\ does this in several steps. First, it computes the physical properties of all the stars present user-specified times using a set of stellar evolutionary tracks. Second, it uses these physical properties to compute the composite spectra produced by the stars, using a user-specified set of stellar atmosphere models. Third and finally, it computes photometry for the stellar population by integrating the computed spectra over a set of specified photometric filters. The results are then written to file.

For a cluster simulation, this procedure is applied to the star cluster being simulated at a user-specified set of output times. For a galaxy simulation, the procedure is much the same, but it can be done both for all the stars in the galaxy taken as a whole, and individually for each star cluster that is still present (i.e., that has not been disrupted). Again, the results are written to file.

\subsection{Monte Carlo Simulation}

The steps described in the previous two simulations are those required for a single realization of the stellar population. However, the entire point of \slug\ is to repeat this procedure many times in order to build up the statistics of the population light output. Thus the entire procedure can be repeated as many times as the user desires, in order to build up statistics.

\section{Compiling and Installing \slug}

\section{Parameter Specification}
\label{sec:parameters}

\section{Probability Distribution Functions in \slug}
\label{sec:pdfs}

\end{document}